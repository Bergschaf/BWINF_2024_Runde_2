\documentclass[a4paper,10pt,ngerman]{scrartcl}
\usepackage{babel}
\usepackage[T1]{fontenc}
\usepackage[utf8x]{inputenc}
\usepackage[a4paper,margin=2.5cm,footskip=0.5cm]{geometry}

% Die nächsten drei Felder bitte anpassen:
\newcommand{\Aufgabe}{Aufgabe 1: \LaTeX-Dokument} % Aufgabennummer und Aufgabennamen angeben
\newcommand{\TeilnahmeId}{?????}                  % Teilnahme-ID angeben
\newcommand{\Name}{Vor- und Nachname}             % Name des Bearbeiter / der Bearbeiterin dieser Aufgabe angeben


% Kopf- und Fußzeilen
\usepackage{scrlayer-scrpage, lastpage}
\setkomafont{pageheadfoot}{\large\textrm}
\lohead{\Aufgabe}
\rohead{Teilnahme-ID: \TeilnahmeId}
\cfoot*{\thepage{}/\pageref{LastPage}}

% Position des Titels
\usepackage{titling}
\setlength{\droptitle}{-1.0cm}

% Für mathematische Befehle und Symbole
\usepackage{amsmath}
\usepackage{amssymb}
\usepackage{amsthm}

% Für Bilder
\usepackage{graphicx}

% Für Algorithmen
\usepackage{algpseudocode}

% Für Quelltext
\usepackage{listings}
\usepackage{color}
\definecolor{mygreen}{rgb}{0,0.6,0}
\definecolor{mygray}{rgb}{0.5,0.5,0.5}
\definecolor{mymauve}{rgb}{0.58,0,0.82}
\lstset{
  keywordstyle=\color{blue},commentstyle=\color{mygreen},
  stringstyle=\color{mymauve},rulecolor=\color{black},
  basicstyle=\footnotesize\ttfamily,numberstyle=\tiny\color{mygray},
  captionpos=b, % sets the caption-position to bottom
  keepspaces=true, % keeps spaces in text
  numbers=left, numbersep=5pt, showspaces=false,showstringspaces=true,
  showtabs=false, stepnumber=2, tabsize=2, title=\lstname
}

\newtheorem{definition}{Definition}



% Diese beiden Pakete müssen zuletzt geladen werden
%\usepackage{hyperref} % Anklickbare Links im Dokument
\usepackage{cleveref}

% Daten für die Titelseite
\title{\textbf{\Huge\Aufgabe}}
\author{\LARGE Teilnahme-ID: \LARGE \TeilnahmeId \\\\
	    \LARGE Bearbeiter/-in dieser Aufgabe: \\ 
	    \LARGE \Name\\\\}
\date{\LARGE\today}

\begin{document}

\maketitle
\tableofcontents

\vspace{0.5cm}

\textbf{Anleitung:} Trage oben in den Zeilen 8 bis 10 die Aufgabennummer, die Teilnahme-ID und die/den Bearbeiterin/Bearbeiter dieser Aufgabe mit Vor- und Nachnamen ein.
Vergiss nicht, auch den Aufgabennamen anzupassen (statt "`\LaTeX-Dokument"')!

Dann kannst du dieses Dokument mit deiner \LaTeX-Umgebung übersetzen.

Die Texte, die hier bereits stehen, geben ein paar Hinweise zur
Einsendung. Du solltest sie aber in deiner Einsendung wieder entfernen!

\section{Lösungsidee}
Jede Codetabelle, bei der $r$ Perlen mit dem selben Durchmesser verwendet werden dürfen, kann als Baum, bei dem jeder Knoten höchstens $r$ Kinder hat, dargestellt werden. ... \\

Im Aufgabenteil b) haben die Perlen aber unterschiedliche, ganzzahlige Durchmesser $c_1 \le c_2 \le \dots \le c_r = C$. Um die Eigenschaft zu erhalten, dass die Länge des Pfads von der Wurzel des Baums zu einem Blatt der Länge des Codes entspricht, müssen die Kanten des Baums der Länge der Perlen entsprechen. ... \\
TODO Beispiel \\

\begin{definition}
    $d_1, d_2, \dots, d_C$ bezeichnet die Anzahl der Perlen mit den Durchmessern $1, 2, \dots, C$.
\end{definition}

\begin{definition}
    Voller Baum (alle Knoten haben entweder 0 oder $r$ Kinder) \\
\end{definition}



Einführung ... \\
Labeling ist trivial (und kann ab jetzt weggelassen werden), weil... \\
\begin{definition}
    Wir bennen die Blätter des Baums nach aufsteigender Tiefe:
    $depth(v_1) \le depth(v_2) \le \dots \le depth(v_n)$
\end{definition}

\begin{definition}
    Wir nennen einen Baum Ebene-$i$-Baum, wenn alle internen Knoten auf einer Ebene $\le i$ liegen. \\
\end{definition}

\begin{definition}
    Die Signatur einer Ebene $i$ des Baums $T$ ist das $C + 1$-Tupel \[sig_i(T) = (m;l_1,l2_, \dots, l_C)\]
wobei $m = |\{v \in T | v ist ein Blatt, tiefe(v) \le i\}|$ die Anzahl der Blätter von T mit einer Tiefe von höchstens $i$ ist und
    \[l_k = |\{u \in T | tiefe(u) = i + k\}, k \in \{1,\dots,C\}\]
die Anzahl der Knoten auf Ebene $i + k$ ist.
\end{definition}
Nun definieren mit die Expand-Operation, die einen Ebene-$i+1$-Baum aus einem Ebene-$i$-Baum erzeugt. \\
\begin{definition}
    Sei $T$ ein Ebene-$i$-Baum mit der Signatur $sig_T = (m;l_1,l_2,\dots,l_C)$. Die Expand Operation wandelt $0 \le q \le l_1$ Blätter auf Ebene $i + 1$ in interne Knoten um, indem $r$ Kinder an jedes dieser Blätter angehängt werden. Die Signatur des resultierenden Ebene-$i+1$-Baums ist
    \[sig_{T'} = (m + l_1, l_2, \dots, l_C) + q * (-1, d_1, d_2, \dots, d_C)\]
\end{definition}
(TODO jeder volle Binärbaum der höhe $h$ kann durch $h$ Expand Operationen erzeugt werden?) \\

Um einen optimalen Code zu finden, definieren wir die Kosten eines Baumes.
\begin{definition}
    Sei $T$ ein Ebene-$i$-Baum mit der Signatur $sig_T = (m;l_1,l_2,\dots,l_C)$. Falls $m \le n$, dann sind die Kosten von $T$ gleich ...


\begin{definition}
    Die Kosten eines Baumes $T$ mit $\ge n$ Blätter sind
    \[cost(T) = \sum_{i=1}^{n} v_i * p_i\]
    (TODO diese Definition geht vom optimalen labeling aus)
\end{definition}

\begin{definition}







\section{Umsetzung}
Hier wird kurz erläutert, wie die Lösungsidee im Programm tatsächlich umgesetzt wurde. Hier können auch Implementierungsdetails erwähnt werden.
TODO erweiterung: die Kosten der Perlen sind keine ganzen Zahlen mehr, sondern reelle Zahlen.
\section{Beispiele}
Genügend Beispiele einbinden! Die Beispiele von der BwInf-Webseite sollten hier diskutiert werden, aber auch eigene Beispiele sind sehr gut – besonders wenn sie Spezialfälle abdecken. Aber bitte nicht 30 Seiten Programmausgabe hier einfügen!

\section{Quellcode}
Unwichtige Teile des Programms sollen hier nicht abgedruckt werden. Dieser Teil sollte nicht mehr als 2–3 Seiten umfassen, maximal 10.



\end{document}
