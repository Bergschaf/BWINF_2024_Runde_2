%! Author = bergschaf
%! Date = 13.01.25

% Preamble
\documentclass[11pt]{article}

% Packages
\usepackage{amsmath}

% Document
\begin{document}

\section{Idee a)}
TODO dokumentation vorlage verwenden
Huffman Codierung (TODO nochmal nachschauen, aber auf Wikipedia steht, dass die optimal ist)\\
oder Shannon-Fano-Kodierung \\
    Huffmann ist besser als Shanon-Fano \\

\section{Idee b)}
Huffmann n-nary Baum, (TODO braucht man das?? -> Eine Kugel hat die Breite 1 (bzw alle andere Kugeln sind teilbar durch
das Maß )) Wenn eine Kugel 2 breit ist, dann ist der Ast vom Binärbaum "länger", er reicht eine Ebene tiefer
https://en.wikipedia.org/wiki/Asymmetric_numeral_systems
    TODO greedy anschaue


\end{document}