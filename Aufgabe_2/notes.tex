%! Author = bergschaf
%! Date = 1/23/25

% Preamble
\documentclass[11pt]{article}

% Packages
\usepackage{amsmath}

% Document
\begin{document}

\şection{Idee}
   Optimale Pfade für beide Labyrinthe berechnen.
    An jeder Position die Bewegungen berechnen, die gegen die Wand führen würden,
    damit dann beide Pfade zusammenführen.
    Zuerst mit der Einschränkung, dass wir Gruben immer vermeiden.

    TODO eigene zufällige labyrinthe mit variabler Dichte

\section{struktur}
    \subsection{Nicht optimale Lösung}
    Zuerst optimale Pfade von beiden labyrinthen berechnen.
    Die Instructions für die einzelnen Schritte speichern. Für jeden Schritt die Bewegungen berechnen, die gegen die
    Wand führen würden. Diese Bewegungen speichern und dann die Pfade zusammenführen.

    TODO annahne das Labyrinth ist von Wänden umgeben

    \section{Modellierungsversuche}
    Graph, jeder Knoten ist ein Tupel aus Positionen, jede Kante ist eine Bewegung.
Wir brauchen einen Pfad von ((0,0),(0,0)) zu ((n-1,m-1),(n-1,m-1)).
von ((a,b),(c, d)) gibt es eine Kante für jede Richtung, in die man gehen kann (oben unten links rechts). Die
    entsprechenden positionen hängen davon ab, ob man gegen eine Wand läuft oder nicht (oder sogar in ein Loch fällt).

\subsection{Laufzeit}
nicht ganz optimal (hängt von der Implementierung ab (TODO auf Wikipedia  von Dijkstra schauen, Fibonacci heaps und
so)),
Obergrenze: $O(V^2)$. Es gibt $n^2 \cdot m^2$ Knoten (glaub i mal). Für jeden Knoten gibt es maximal vier Kanten.
Also ist die Laufzeit ....
TODO schauen ob der Algorithmus überhaupt stimmt
TODO falls der stimmt noch eine clevere heuristik finden


TODO Annahme: Gruben teleportieren (und man muss nicht extra laufen)


    \section{NP schwer?}

\end{document}